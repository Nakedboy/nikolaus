
\documentclass{studArbeit}

%Trennungsregeln
\hyphenation{Beispiel}



\begin{document}
%%%Hier alles wichtige ausf�llen und schon werden Titelseite und Aufgabenstellung zusammengebaut
%Tragen Sie hier alle Daten f�r die TitelSeite ein:
\ArtDerArbeit{Studienarbeit}
\AusgabeArbeit{15.04.2013}
%Der Titel der Arbeit in Deutsch und Englisch
\TitelDeutsch{Analyse und Vergleich von Methoden zur Bestimmung\\[2mm] des Zusatzwiderstandes im Seegang}
\TitelEnglisch{Analysis and comparison of different Methods\\ for added resistance in seaways} 
%Hier kommt die Aufgabenstellung herein
\TextAufgabenstellung{
Die Bestimmung des Widerstandes und damit der Antriebsleistung einer Schiffsform ist im Entwurfsprozess von hoher Bedeutung. In der Regel handelt es sich bei dem dabei ermittelten Wert allerdings nur um den Galttwasserwiderstand des Schiffes, w�hrend zus�tzlicher Widerstandsanteile - zum Beispiel bedingt durch Seegang - �ber einen pauschalen Aufschlag in H�he von 10 - 15\% der sogenannten Sea-Margin, ber�cksichtigt werden. Im Zuge der anhaltenden Wirtschaftskrise und dem darausfolgendem Druck zur Entwicklung effizienter Schiffsformen wird derzeit vermehrt daran gearbeitet, diesem recht ungenauen Ansatz pr�zisere Methoden entgegenzusetzen.\\
\textbf{Ziel der Arbeit} ist es, eine Analyse der zur Zeit zur Verf�gung stehenden Methoden zur Ermittlung der umweltbedingten Zusatzwiderst�nde durchzuf�hren. Dabei sollen sowohl �berschl�gige als auch numerische Verfahren untersucht werden. Darauf aufbauend sollen ausgew�hlte Methoden implementiert und anhand eines Beispielschiffes miteinander verglichen werden.\\
Die im Rahmen der Arbeit durchzuf�hrenden T�tigkeiten und Aufgabenpakete sind sp�tenstens vier Wochen nach Bearbeitungsbeginn mit dem Betreuer abzustimmen. Daf�r ist ein Kurzexpos\'{e} der Arbeit vorzulegen.
}
%Autor
\Autor{Christopher Leu}
\AutorMatrikel{003202036}
\AutorEmail{christopher.leu@uni-rostock.de}
%Studiengang: Master Maschinenbau / Master Schiffs- und Meerestechnik
\AutorStudiengang{Master Schiffs- und Meerestechnik}
\Bearbeitungszeitraum{6 Monate}
%Gutachter der Arbeit
\ErstGutachter{Prof. Dr.-Ing. Robert Bronsart}
\ZweitGutachter{Dipl.-Ing. Jonas Wagner}


%Ab hier nichts ver�ndern
% Tragen Sie die verwendeten Formeln und Abk�rzungen in die datei "`Include/Akb-Formel.tex"` in die entsprechenden Tabellen ein.
% Ihre Arbeit schreiben Sie in dem Dokument "'Include/main.tex"`
% Abbildungen, Tabellen, etc. sollten Sie in das Verzeichnis "'Inlcude/figures"` speichern.
% Ihre Literaturquellen speichern Sie in der Datei "'Include/Bib/bib.bib"` ab.
\automark[]{chapter}
\titelSeite
\aufgabenstellung
\toc
\pagestyle{scrplain}
\chapter*{Abk�rzungsverzeichnis}
\begin{acronym}[l�ngstes Akronym]
\setlength{\parskip}{0.7em} 
\acro{K�rzel}{Bedeutung}{: Erkl�rung}
\acro{TEU}{Twenty-foot Equivalent Unit}{: standardisierte Einheit zur Z�hlung von genormten Containern}
\end{acronym}
\clearpage

\chapter*{Formelverzeichnis}


\begin{longtable}{>{\centering$}p{1.6cm}<{$}>{\centering$}p{1.7cm}<{$}p{11.4cm}}
	\textbf{Symbol} & \textbf{Einheit} & \textbf{Bedeutung}
	\endfirsthead
	\textbf{Symbol} & \textbf{Einheit} & \textbf{Bedeutung}
	\endhead
%Ab hier die Formelzeichen ihre Einheit und Bedeutung aufschreiben
\end{longtable}

Schiffsparameter:\\

\begin{longtable}{>{\centering$}p{1.6cm}<{$}>{\centering$}p{1.7cm}<{$}p{11.4cm}}

L_{pp}		& [m]		& L�nge zwischen den Loten\\
L_{oa}		& [m]		& L�nge �ber alles\\
B 			& [m]		& Breite\\
T 			& [m]		& Tiefgang\\
\nabla 		& [m^3]		& Verdr�ngung des Schiffes\\
cb			& [-]		& Blockkoeffizient\\
Fr 			& [-]		& Froude-Zahl\\
V_S			& [kn]		& Geschwindigkeit des Schiffes\\
C_{S}		& [-]		& Widerstandsbeiwert des Schiffes\\

\end{longtable}

Zus�tzlicher Widerstand in Wellen:\\

\begin{longtable}{>{\centering$}p{1.6cm}<{$}>{\centering$}p{1.7cm}<{$}p{11.4cm}}


R 			& [N]		& Gesamtwiderstand\\
R_G			& [N]		& Glattwasserwiderstand\\
R_{AW}		& [N] 		& Widerstand durch Wellen\\
R_{AS}		& [N] 		& durch �nderung der Temperatur und Wasserdichte\\
R_{B}		& [N]		& durch Bewuchs\\
C_{AW}		& [-]		& Widerstandsbeiwert bei Widerstand durch Wellen\\
R_{wave}	& [N]		& Widerstand durch regul�rem Seegang\\

\end{longtable}


Zus�tzlicher Widerstand durch den Wind:\\

\begin{longtable}{>{\centering$}p{1.6cm}<{$}>{\centering$}p{1.7cm}<{$}p{11.4cm}}

R_{AA}		& [N] 		& durch Wind\\
C_{AA}		& [-]		& Widerstandsbeiwert Wind\\
V_{WR}		& [m/s]		& realtive Windgeschwindigkeit\\
\rho_{A}	& [kg/m^3]	& Dichte der Luft\\
\psi_{WR}	& [rad]		& Windrichtung\\
A_{XV}		& [m^2]		& Windangriffsfl�che\\
\mu			&			& Reduktionsfaktor Windrichtung\\

\end{longtable}


Wellenmodell:\\

\begin{longtable}{>{\centering$}p{1.6cm}<{$}>{\centering$}p{1.7cm}<{$}p{11.4cm}}
	\textbf{Symbol} & \textbf{Einheit} & \textbf{Bedeutung} \\[3ex]
	\endfirsthead
	\textbf{Symbol} & \textbf{Einheit} & \textbf{Bedeutung} \\[3ex]
	\endhead

\zeta_A		& [m]		& Wellenamplitude\\
g 			& [m/s^2]	& Erdbeschleunigung\\
c 			& [m/s]		& Wellenfortschrittsgeschwindigkeit\\
\rho 		& [kg/m^3]	& Massendichte\\
\omega		& [rad/s]	& Wellenfrequenz\\
k 			& [rad/m]	& Wellenzahl\\
\lambda 	& [m]		& Wellenl�nge\\
\alpha		& [\circ]	& Winkel zwischen der Schiffsl�ngsachse und der Richtung einkommender Wellen\\
E 			& 			& Richtungsspektrum\\
H_{Hw1/3}	& [m]		& signifikante Wellenh�he\\
S_f			& [-]		& Frequenzspektrum\\
m_n			& [Nm]		& n-te Moment des Frequenzspekrtum\\


\end{longtable}
\clearpage{}
\start

%Datei mit Haupt-Text
\includefrom{Include/}{main}
%Anhang
\appendix
\includefrom{Include/}{Anhang}
%Bibliography
\bibliographystyle{unsrt} 
\bibliography{Include/Bib/bib} %Ort der Bib-Tex-Datei
\selbst
\end{document}
