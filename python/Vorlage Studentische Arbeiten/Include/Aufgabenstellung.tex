% and here we go ---------------------- 1. page of text --------------
\thispagestyle{empty}
\setfootsepline{}[\color{white}]
\setheadsepline{}[\color{white}]

\vspace*{-20mm}% vergr��ern nach oben -20mm
\enlargethispage{5mm}% vergr��ern nach unten
\setlength{\parindent}{0mm}

\vspace*{10mm}%
\begin{center}
{\Large%
Analyse und Vergleich von Methoden zur Bestimmung des Zusatzwidertandes im Seegang
\\[5mm]%
\normalsize%
Studienarbeitarbeit\\[2mm]
Frau Christopher Leui\\
Matrikelnummer: 3202036\\[10mm]
}%
\end{center}

Die Optimierung von Schiffsentw�rfen auf mehr als nur einen Betriebspunkt hat unter anderem durch die Weltwirtschaftskrise in den letzten Jahren stark an Bedeutung gewonnen. So werden moderne Schiffe in der Zwischenzeit auf mehrere Kombinationen aus Schwimmlagen und Geschwindigkeiten ausgelegt.\\ 
\textbf{Ziel der Arbeit} ist es, eine parametrische Optimierung eines Containerschiffes in Bezug auf die Schleppleistung durchzuf�hren. Die Zielfunktion dieser Optimierung soll dabei aus einer Kombination mehrerer unterschiedlich gewichteter Betriebszust�nde (Tiefgang, Geschwindigkeit) abgeleitet werden.\\
Die Basis dazu bilden statistische Daten von bereits in Fahrt befindlichen Schiffen, welche durch den Betreuer zur Verf�gung gestellt werden. Aus diesen Daten ist das zuk�nftige Betriebsprofil des zu untersuchenden Schiffes abzuleiten, wobei Unsicherheiten bei der Prognose mit Hilfe von Verteilungsfunktionen zu ber�cksichtigen sind. Durch eine statistische Analyse aller so entwickelten Betriebszust�nde soll darauffolgend die Zielfunktion - bestehend aus drei bis f�nf ihrer Auftretensh�ufigkeit nach gewichteten Betriebszust�nden - abgeleitet werden.\\
\textbf{Ablauf der Arbeit:} Nach der Einarbeitung in das Thema ist ein Expos� zu erstellen, welches eine genauere Beschreibung des L�sungsweges sowie einen zugeh�rigen Zeitplan enth�lt. Diese ist sp�testens vier Wochen nach Ausgabe der Aufgabenstellung mit dem Betreuer zu diskutieren und bildet die Grundlage f�r das weitere Vorgehen in der Bearbeitung der Aufgabenstellung.\\
\\
\\
\\
\\
\\
\\
\\
\\
\\
\\
\begin{tabbing}
Ausgabe der Arbeit:  \= 21.05.2012 \kill%
Betreuer: \> Dipl.-Ing. Jonas Wagner; Prof. Dr.-Ing. Robert Bronsart\\
Ausgabe der Arbeit: \> 21.05.2012\\
Abgabe der Arbeit: \> 21.09.2012\\
\end{tabbing}





\setfootsepline{}[\color{black}]
\setheadsepline{}[\color{black}]